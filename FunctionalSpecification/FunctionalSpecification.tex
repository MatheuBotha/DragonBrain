\documentclass[11pt]{article}
\usepackage{graphicx}
\graphicspath{ {images/} }
\begin{document}

\begin{titlepage}

\begin{center}
\begin{huge}
Swarm Visualiser - COS 301 Main Project
\\
Functional Specification
\begin{small}
\\
Team: Dragon Brain
\\
Members:
\\
Matheu Botha u14284104
\\
Renton McInytre u14312710
\\
Emilio Singh u14006512
\\
Gerard van Wyk u14101263

\end{small}

\end{huge}
\end{center}
\end{titlepage}

\pagebreak

\tableofcontents

\pagebreak
\section{System Domain Model}
\paragraph{•}
In this section we will discuss and present a system-wide domain model and present the constituent systems, Modules, and their scopes

\begin{figure}[h]

\end{figure}
\section{Modules}
\subsection{General Optimiser}
\paragraph{•}
The General Optimiser or OPT for short is the module that provides the problem solving capacities of the system. It makes use of swarm-based methods to traverse user defined search spaces and perform evaluations of particles within the search space against user defined objective functions.
\subsubsection{Scope}
\paragraph{•}
The scope of the General Optimiser is presented below.
\begin{figure}[h]
\includegraphics[scale=0.45]{Scope.png}
\end{figure}

\paragraph{•}
The User has 3 areas of general purpose use
\begin{itemize}
\item Creating Settings Packages which configure the Optimiser for runs
\item Changing Optimiser parameters during a run
\item Presenting problems for the Optimiser to solve
\end{itemize}
\subsubsection{Service Contracts}
\paragraph{•}
We will now specify the service contracts that importantly define the capacities of the General Optimiser Module

\subsubsection{Create Optimiser}
\begin{figure}[h]
\includegraphics[scale=0.4]{serviceContractOPT.png}
\end{figure}

\paragraph{•}
The user will request for an Optimiser to be created. If there is space, that is less than 4 Optimisers already exist in the System, then the user will construct, indirectly, a SettingsPackage by entering their configuration parameters for the Optimiser. This Settings Package will be used by the Manager to construct and configure an Optimiser for use.

The exact conditions will be codified below:
\begin{enumerate}
\item There must be fewer than 4 Optimisers currently running in the system
\item The SettingsPackage received must contain specifications for all configuration parameters and must contain values for those parameters within legal domains.
\item The Manager Module is ready to receive/able to receive a new order/request from the User Interface.
\end{enumerate}

\subsubsection{Functional Requirements}
\begin{figure}[h]
\includegraphics[scale=0.5]{functionalRequirementsOPT.png}
\end{figure}

\paragraph{•}
In terms of the functional requirements for the creation of an Optimiser object, a number of factors need to be considered in order for this to realised. Firstly, the reading and consumption of a Settings Package object relates to the process of extracting the necessary information from the settings package, the GenOPT package components, in order to provide the information during a constructor call. During the process, the Settings Package received is now invalidated and must be destroyed as it is at the end of its life-cycle. Once configuration has been completed, the GenOpt object will be produced and used by the Manager object.
\subsubsection{Process Design}
\paragraph{•}
The process specification defined here is for the process to create an Optimiser object.
\begin{figure}[h]
\includegraphics[scale=0.5]{GeneralOptimiserActivity.png}
\end{figure}

\subsubsection{Domain Model}
\paragraph{•}
Presented below is the domain model of the General Optimiser.
\begin{figure}[h]
\includegraphics[scale=0.2]{OPTModel.png}
\end{figure}

\subsection{Graphics Processor}
\subsubsection{Scope}
\subsubsection{Service Contracts}
\subsubsection{Functional Requirements}
\subsubsection{Process Design}

\subsection{Manager}
\subsubsection{Scope}
\subsubsection{Service Contracts}
\subsubsection{Functional Requirements}
\subsubsection{Process Design}

\subsection{User Interface}
\subsubsection{Scope}
\subsubsection{Service Contracts}
\subsubsection{Functional Requirements}
\subsubsection{Process Design}
\end{document}
