\documentclass[12pt]{article}
\usepackage{graphicx}
\begin{document}
\title{Document Tender}

\begin{titlepage}
\begin{huge}
\begin{center}
Project Tender

Project: STORM
\\
\begin{LARGE}
Client: Linda Marshall \& Vreda Pieterse
\end{LARGE}

Team: Dragon Brain
\\
\begin{small}
Department of Computer Science, University of Pretoria
\\
\begin{itemize}
\item Matheu Botha u14284104
\item Renton McIntyre u14312710
\item Emilio Singh u14006512
\item Gerard van Wyk u14101263
\end{itemize}


Date: 2016/05/02

\end{small}

\includegraphics[scale=0.2]{test}
\end{center}

\end{huge}


\end{titlepage}

\pagebreak

\section{The Team}
\subsection{Matheu Botha}
\includegraphics[scale=0.02]{a}
\paragraph{Interests}
My main interests lie in Artificial Intelligence. This is applicable to the project as pattern recognition is largely considered one of the more difficult areas in computer science and some of the most progressive solutions have come from artificial intelligence algothrithms.
\paragraph{Technical Skills}
	I am proficient in the following programming languages:
	\begin{itemize}
		\item C++
		\item C#
		\item Java
		\item Javascript
		\item Python
		\item PHP
	\end{itemize}

\paragraph{Relevant Past Experience}
	In the past I have done projects with Monkey and River developing the SALGA Municipal Barometer. I have also worked with Merlynn Intelligent Technologies developing the UP2TOM app. Both of these projects involved extensive Front End development which has given me experiance with regard to developing User Interfaces, which is relevant for the project. I have also been working as a part time technical assistant for the university of pretoria which has afforded me the opportunity to work with many enterprise level development technologies in a production environmet.

\paragraph{Non-technical Strengths} 
	I enjoy working as part of a team on a large project. I am also very challenge driven. I enjoying taking on challenging projects and working hard with a group of like minded individuals to overcome those challenges.

\paragraph{Motivation for Project}
	I would like to work on this project because it is a very progressive field that still has the potential for a lot of innovative work to be done in it.

\subsection{Renton McIntyre}
\includegraphics[scale=0.25]{Renton}
\paragraph{Interests}
My relevant interests include the fields of Artificial Intelligence and general academic advancement and experimentation.
\paragraph{Technical Skills}
I have had prolonged exposure to Java and C++ (the latter of which I consider to be my primary language). I also have experience with various scripting languages, C and Assembly.

Otherwise, I have some mathematical prowess and, perhaps above all else, a proficiency in the ability to learn technical skills on short notice.
\paragraph{Relevant Past Experience}
I have very few past experience with such projects. However, I am willing to learn and believe I will be able to make a valuable contribution towards the project.

\paragraph{Non-technical Strengths}
I believe I have a strong capability to work with people and have previously shown my ability to maintain my connection with a team well, often from the perspective of serving as somewhat of an assistant to the person in charge, as I do not often take such a role myself, despite potentially having the capability. 
\newline I also believe I have a good ability to apply concepts to practicality.
\paragraph{Motivation for Project}
The STORM project holds interest for me largely from an academic perspective. I would appreciate the chance to work with the University from an internal perspective, get to know the inner workings of the system and simultaneously provide a service to help the University and future students in a constructive manner. 

\subsection{Emilio Singh}
\includegraphics[scale=0.2]{Emilio}
\paragraph{Interests}
My relevant interests include the study of cryptography, video games and reading.
\paragraph{Technical Skills}
I can program using Java, C++, and a variety of scripting languages proficiently.
Furthermore, I have exposure, and somewhat fluency in Assembly.

My true strength would be in the field of mathematics and analysis however as I have shown high level competencies in various mathematical subjects while at university.
\paragraph{Relevant Past Experience}
Other than the 301 Mini Project, I have not worked on any major software system to this caliber but rest assured, I am very eager to learn and contribute towards the project.
\paragraph{Non-technical Strengths}
In terms of non-technical strengths, I would say that I have a capacity to serve in a leadership role. I have proven on numerous occasions that I can work with others and coordinate them towards a particular aim.

Furthermore, I feel that I have a level of skill with regards to communication with people from both technical and non-technical backgrounds which would typically aid in client-team communications.
\paragraph{Motivation for Project}
The appeal of the STORM project is in very much the same class as the appeal of pulling back the curtain to reveal the Wizard of Oz. The STORM project offers me a chance to work with the university in a way I have not before to actually do something that would have an appreciable effect on the lives of my fellow students.

\subsection{Gerard van Wyk}
\includegraphics[scale=0.02]{a}
\paragraph{Interests}
I am highly interested in Artificial Intelligence, video games/virtual reality and reading about science(especially space, geology and natural selection).
\paragraph{Technical Skills}
My programming language proficiencies are in Java, C++, and to a lesser extent OpenGL and general web languages.
Aside from programming languages: I am good at mathematics, and have a moderate skill in image and video editing.
\paragraph{Relevant Past Experience}
I completed the COS 301 mini-project stage and have worked on simple databases.
\paragraph{Non-technical Strengths}
I work well with other people, learn quickly, am driven by intellectual curiosity, and always seek to achieve an efficient solution.
\paragraph{Motivation for Project}
By working on STORM I get the opportunity to closely work with the university on a project that will improve the lives of next year's 3rd year COS students, and gather valuable software development experience.

\section{Project Execution}
\paragraph{Developmental Methodology}
We will be following an Agile development methodology. We hope to realise this development methodology by doing the following:
\begin{enumerate}
\item Communicating with our clients to ensure they are informed of project progress as well as to constantly refine and update our understanding of requirements in order to meet them.
\item Construct a frequent, working software delivery schedule so that incrementally working segments of the STORM project can be delivered to our clients.
\item Commit to the delivery of simplistic, maintainable and efficient software.
\item Accept that requirements for the project may change constantly throughout the project life-cycle and develop a modular system whose functioning can be easily adapted to suit this purpose.
\item We have a strong commitment to compartmentalisation of work and will adopt pair-programming and other team-oriented development practices as we are able to.
\end{enumerate}

\paragraph{Client-Team Communication Methods}
The clients of this project are both members of the CS department. As such, bi-weekly, on-site meetings with a face-to-face communication process between all team members and clients will be the primary means of communication.

Furthermore, our team maintains an open door policy in that our contact details will also be available to the clients so that we may be emailed any additional concerns or requests outside of the traditional meeting time. We also hope to have a similar capacity in terms of contacting our clients outside of normal operating hours should the need so arise.

Finally, we have also created a Slack group to manage the project team for the duration of the project. The clients will be extended the offer to join the group to be more directly related to intra-team communications.
\paragraph{Initial Ideas in response to Technical Challenges}
Based on the project vision and objectives, there are two main technical challenges that need to be overcome.

The first is to design a system that automatically performs team building. The second is to create a functionality within the system to aid in data analysis for the user so that less time is wasted on compiling data for analysis.

To the second problem, it would be worthwhile to consider researching data analytics engines, either to write our own or make use of such a component provided by an external vendor in aiding the process of data analysis.

To the first problem, developing an automated match making system would require an algorithm that, in theory, would be modeled after the various kinds of social networking algorithms that are used by programs like Facebook to suggest friends based off compatibility scores. This service would have to first define a set of characteristics for each person and then use those characteristics to construct a profile around which other people are considered for matches for teams.
\paragraph{Potential Technologies for use where not specified by client}
One major potential consideration for further investigation is the Google Analytics API, an API provided by Google for the purposes of data analysis and collection. Integrating the API into a software service would provide a considerable tool towards realising the goals of the project.
\paragraph{Deliverable to Client}
The STORM system will be presented as a single software package that can be accessed via any of the major browsers that supports the ability to automatically determine teams for COS 301 Mini Project "Rocking the Boat" and for some data analysis and gathering capacity for the users.
\end{document}
