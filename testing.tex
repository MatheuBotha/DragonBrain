\documentclass[11pt]{article}
\usepackage{graphicx}
\usepackage{lipsum}
\graphicspath{ {images/} }
\begin{document}

\begin{titlepage}

\begin{center}
\begin{huge}
Swarm Visualiser - COS 301 Main Project
\\
Testing Specifications
\begin{small}
\\
Team: Dragon Brain
\\
Members:
\\
Matheu Botha u14284104
\\
Renton McInytre u14312710
\\
Emilio Singh u14006512
\\
Gerard van Wyk u14101263

\end{small}

\end{huge}
\end{center}
\end{titlepage}

\pagebreak

\tableofcontents

\section{Frameworks and Mechanisms}
\subsection{Testing Framework}
\lipsum
\subsection{Unit Tests}
\lipsum
\section{Testing Strategies}
\subsection{Unit Tests}
\lipsum
\section{Integration Tests}
\paragraph{•}
This section of the testing manual will contain all of the necessary details pertaining to the area of Integration Tests, and Integration Testing, as per project requirements and project dictates.
\subsection{Integration Requirements}
\paragraph{•}
In terms specifying the Integration Requirements for the project, we identify Program Modules. These Modules are separate components that will each provide services and possibly make use of other services from other Modules. As an integration challenge, the task is to ensure that all of these Modules are able to realise their service contracts and do not contribute to the failure of service contracts of other Modules.

The Modules are:
\begin{itemize}
\item Graphics Pipeline: The Graphics Pipeline will be required to integrate with the Manager and Data Models Module. This integration will take the form of message passing,System Snapshots, which are consumed by the Pipeline as provided by the Manager. The Graphics Pipeline similarly affects the User Interface. The Graphics Pipeline will then make use of the system snapshot in order to render, on the User Interface, the implications realised internally by the General Optimiser.
\item General Optimiser: The General Optimiser will produce System Snapshots which will be stored in Data Models inside the Manager Module. This General Optimiser, by the use of Settings Package, will then be configured to meet specific user requirements. 
\item User Interface: The User Interface is going to communicate with the Manager Module. The user changing system parameters, or configuring them, will generate Settings Package objects and these will be sent to the Manager for use in adjusting configurations. The User Interface will also be receiving, and rendering, message packets from the Graphics Pipeline. These packets will result in new visual information being displayed to the user.
\item Manager \& Data Models:
\end{itemize}
 
\subsection{Strategies}

\section{Areas of Concern}
\subsection{Key Points of Failure}
\lipsum

\section{Example Unit Test}

\end{document}
